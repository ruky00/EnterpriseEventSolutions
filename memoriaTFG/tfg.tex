% Clase del documento
\documentclass[12pt,twoside,titlepage]{report}





%%%%%%%%%%%%%%%%%%%%%%% Paquetes %%%%%%%%%%%%%%%%%%%%%%%

\usepackage[a4paper,inner=3cm,outer=2cm,top=3cm,bottom=3cm,left=2.5cm,right=2.5cm,bindingoffset=5mm]{geometry}


% Usad \usepackage[dvips]{graphicx} o \usepackage[pdftex]{graphicx} (no ambos)
%\usepackage[dvips]{graphicx} %%% para LaTeX. Las figuras deben estar en formato eps

\usepackage[colorlinks=true,pdftex]{hyperref}   %%% Opcional. Para incluir marcadores y enlaces en el pdf
\usepackage[pdftex]{graphicx}  %%% para pdflatex. Las figuras pueden estar en pdf, jpg, svg y otros formatos


\usepackage[spanish]{babel}

%\usepackage[latin1]{inputenc} % Usad en WinEdt/MikTex
\usepackage[utf8]{inputenc} % Usad en overleaf

%\usepackage[T1]{fontenc}


% Algunos paquetes útiles

\usepackage{amsmath,amssymb}
\usepackage{hyperref}
\usepackage{xcolor}
\usepackage{afterpage}
\usepackage{paralist}
\usepackage{array}
\usepackage{enumerate}
\usepackage{paralist}
\usepackage{enumitem}
\usepackage{float}
\usepackage{setspace}
\usepackage{listings}
\usepackage{algorithm}
\usepackage{algorithmic}
\usepackage{fancyhdr}
\usepackage{rotating}
\usepackage{multirow}
\usepackage{longtable}

% Otros paquetes

\usepackage{quotchap}
\usepackage{lipsum}

%%%%%%%%%%%%%%%%%%%%%%%%%%%%%%%%%%%%%%%%%%%%%%%%%%%%%%%%






%%%%%%%%%%%%%%%%%%%%%%% Definiciones básicas %%%%%%%%%%%%%%%%%%%%%%%

\newcommand{\nombreautor}{Carlos Fernández López}
\newcommand{\nombretutor}{NombreTutor Apellido1 Apellido2}
\newcommand{\titulotrabajo}{Enterprise Event Solutions}
\newcommand{\escuela}{Escuela Técnica Superior\\de Ingeniería Informática}
\newcommand{\escuelalargo}{Escuela Técnica Superior de Ingeniería Informática}
\newcommand{\universidad}{Universidad Rey Juan Carlos}
\newcommand{\fecha}{Fecha}
\newcommand{\grado}{Grado en Ingeniería del Software}
\newcommand{\curso}{Curso 2023-2024}
\newcommand{\logoUniversidad}{logoURJC.pdf} % logoURJC.eps
\newcommand{\logoEVS}{EVS.png} % logoEVS.eps

%%%%%%%%%%%%%%%%%%%%%%%%%%%%%%%%%%%%%%%%%%%%%%%%%%%%%%%%%%%%%%%%%%%%






%%%%%%%%%%%%%%%%%%%%%%%%% Otras definiciones %%%%%%%%%%%%%%%%%%%%%%%%%%

% Definiciones de colores (para hidelinks)
\definecolor{BlueLink}{rgb}{0.165,0.322,0.745}
\definecolor{PinkLink}{rgb}{0.8,0.22,0.5}
\definecolor{gray}{rgb}{0.6,0.6,0.6}


% Enlaces
\hypersetup{hidelinks,pageanchor=true,colorlinks,citecolor=PinkLink,urlcolor=black,linkcolor=BlueLink}


\newcommand\blankpage{%
    \newpage
    \null
    \thispagestyle{empty}%
    %\addtocounter{page}{-1}%
    \newpage}


% Texto referencias
\addto{\captionsspanish}{\renewcommand{\bibname}{Bibliografía}}

% Texto Índice de tablas
\addto\captionsspanish{
\def\tablename{Tabla}
\def\listtablename{\'{I}ndice de tablas}
}


\floatname{algorithm}{Algoritmo}

\newfloat{algorithm}{t}{lop}

%% Etiquetas de comentarios (tutor/alumno)
\newif\ifdraft
\drafttrue
\usepackage{subcaption}
\input{macros}


%\newenvironment{pseudocodigo}[1][htb]
%  {\renewcommand{\algorithmcfname}{Pseudocódig}% Update algorithm name
%   \begin{algorithm}[#1]%
%  }{\end{algorithm}}
  
%%%%%%%%%%%%%%%%%%%%%%%%%%%%%%%%%%%%%%%%%%%%%%%%%%%%%%%%%%%%%%%%%%%%





%%%%%%%%%%%%%%%%%%%%%%% Estilo de código (en Python) %%%%%%%%%%%%%%%%%%%%%%%

\definecolor{bg}{rgb}{0.95,0.95,0.95}
\definecolor{mydeepteal}{rgb}{0.16,0.22,0.23}
\definecolor{myteal}{rgb}{0.31,0.44,0.46}
\definecolor{mymediumteal}{rgb}{0.41,0.58,0.60}

\DeclareFixedFont{\ttb}{T1}{txtt}{bx}{n}{12} % for bold
\DeclareFixedFont{\ttm}{T1}{txtt}{m}{n}{12}  % for normal


%\newcommand*{\FormatDigit}[1]{\textcolor{mydeepteal}{#1}}
\newcommand*{\FormatDigit}[1]{\textcolor{black}{#1}}

% Python style for highlighting
\newcommand\mypythonstyle{\lstset{
language=Python,
basicstyle=\ttfamily\small,
%basicstyle=\linespread{1.0}\footnotesize\ttm,
otherkeywords={self},             % Add keywords here
keywordstyle=\bfseries\ttfamily\color{myteal},
%keywordstyle=\ttb\color{myteal},
commentstyle=\itshape\color{myteal},
stringstyle=\color{mydeepteal},
emph={MyClass,__init__},          % Custom highlighting
emphstyle=\ttb\color{mydeepteal},    % Custom highlighting style
% Any extra options here
showstringspaces=false,            %
backgroundcolor=\color{bg},
rulecolor = \color{bg},
%identifierstyle=\color{deepgreen},
breaklines=true,
numbers=left,
numbersep=5pt,
numberstyle=\tiny,
tabsize=4,
xleftmargin=1em,
frame = single,
framesep = 3pt,
framextopmargin=0pt,
framexbottommargin=0pt,
framexleftmargin=0pt,
framexrightmargin=0pt,
fontadjust=true,
basewidth=0.55em, % compactness of code
upquote=true,
}}

% Python environment
\lstnewenvironment{mypython}[1][]
{
\mypythonstyle
\lstset{#1}
}
{}

\newcommand\mypythonstylenormalinline{\lstset{
language=Python,
basicstyle=\ttfamily\normalsize,
%basicstyle=\linespread{1.0}\footnotesize\ttm,
otherkeywords={self},            % Add keywords here
keywordstyle=\bfseries\ttfamily\color{myteal},
%keywordstyle=\ttb\color{myteal},
commentstyle=\itshape\color{mymediumteal},
stringstyle=\color{mydeepteal},
emph={MyClass,__init__},          % Custom highlighting
emphstyle=\ttb\color{mydeepteal},    % Custom highlighting style
% Any extra options here
showstringspaces=false,            %
backgroundcolor=\color{bg},
rulecolor = \color{bg},
%identifierstyle=\color{deepgreen},
breaklines=false,
numbers=left,
numbersep=5pt,
numberstyle=\tiny,
tabsize=4,
xleftmargin=0em,
frame = single,
framesep = 3pt,
framextopmargin=0pt,
framexbottommargin=0pt,
framexleftmargin=0pt,
framexrightmargin=0pt,
fontadjust=true,
%basewidth=0.55em, % compactness of code
upquote=true,
}}

\newcommand\mypythoninline[1]{{\mypythonstylenormalinline\lstinline!#1!}}

%%%%%%%%%%%%%%%%%%%%%%%%%%%%%%%%%%%%%%%%%%%%%%%%%%%%%%%%%%%%%%%%%%%%%%%%%%%%%%




%%%%%%%%%%%%%%%%%%%%%%%%%%%% Comandos definidos por el autor 

\newcommand{\transpuesta}{\mbox{\tiny $\mathsf{T}$}}

% Java style for highlighting
\newcommand\myjavastyle{\lstset{
language=Java,
basicstyle=\ttfamily\small,
otherkeywords={},             % Add keywords here
keywordstyle=\bfseries\ttfamily\color{blue},
commentstyle=\itshape\color{green},
stringstyle=\color{red},
emph={},
emphstyle=\bfseries\ttfamily\color{blue},
showstringspaces=false,
backgroundcolor=\color{lightgray},
rulecolor = \color{lightgray},
breaklines=true,
tabsize=4,
xleftmargin=1em,
frame = single,
framesep = 3pt,
framextopmargin=0pt,
framexbottommargin=0pt,
framexleftmargin=0pt,
framexrightmargin=0pt,
fontadjust=true,
basewidth=0.55em,
upquote=true,
}}

% Java environment
\lstnewenvironment{myjava}[1][]
{
\myjavastyle
\lstset{#1}
}
{}

\newcommand\myjavastylenormalinline{\lstset{
language=Java,
basicstyle=\ttfamily\normalsize,
otherkeywords={},            % Add keywords here
keywordstyle=\bfseries\ttfamily\color{blue},
commentstyle=\itshape\color{green},
stringstyle=\color{red},
emph={},
emphstyle=\bfseries\ttfamily\color{blue},
showstringspaces=false,
backgroundcolor=\color{lightgray},
rulecolor = \color{lightgray},
breaklines=false,
numbers=left,
numbersep=5pt,
numberstyle=\tiny,
tabsize=4,
xleftmargin=0em,
frame = single,
framesep = 3pt,
framextopmargin=0pt,
framexbottommargin=0pt,
framexleftmargin=0pt,
framexrightmargin=0pt,
fontadjust=true,
upquote=true,
}}

\newcommand\myjavainline[1]{{\myjavastylenormalinline\lstinline!#1!}}



\lstnewenvironment{myvue}[1][]
{
\myvuestyle
\lstset{#1}
}
{}

\newcommand\myvuestyle{\lstset{
language=HTML, % Vue.js is primarily HTML, with embedded JavaScript and CSS
basicstyle=\ttfamily\normalsize,
otherkeywords={data,methods,computed,props,template,script,style,export,default,name,components,created,mounted}, % Add Vue.js specific keywords here
keywordstyle=\bfseries\ttfamily\color{blue},
commentstyle=\itshape\color{green},
stringstyle=\color{red},
emph={},
emphstyle=\bfseries\ttfamily\color{blue},
showstringspaces=false,
backgroundcolor=\color{lightgray},
rulecolor = \color{lightgray},
breaklines=true,
tabsize=2,
xleftmargin=0em,
frame = single,
framesep = 3pt,
framextopmargin=0pt,
framexbottommargin=0pt,
framexleftmargin=0pt,
framexrightmargin=0pt,
fontadjust=true,
upquote=true,
}}
\newcommand\myvueinline[1]{{\myvuestyle\lstinline!#1!}}

\lstnewenvironment{myyaml}[1][]
{
\myyamlstyle
\lstset{#1}
}
{}

\newcommand\myyamlstyle{\lstset{
language=YAML,
basicstyle=\ttfamily\normalsize,
keywordstyle=\bfseries\ttfamily\color{blue},
commentstyle=\itshape\color{green},
stringstyle=\color{red},
showstringspaces=false,
backgroundcolor=\color{lightgray},
rulecolor=\color{lightgray},
breaklines=true,
tabsize=2,
xleftmargin=0em,
frame=single,
framesep=3pt,
framextopmargin=0pt,
framexbottommargin=0pt,
framexleftmargin=0pt,
framexrightmargin=0pt,
fontadjust=true,
upquote=true,
}}

\newcommand\myyamlinline[1]{{\myyamlstyle\lstinline!#1!}}




%%%%%%%%%%%%%%%%%%%%%%%%%%%%%%%%%%%%%%%%%%%%%%%%%%%%%%%%%%%%%%%%%%%%%%%
%                           Inicio del documento                       
%%%%%%%%%%%%%%%%%%%%%%%%%%%%%%%%%%%%%%%%%%%%%%%%%%%%%%%%%%%%%%%%%%%%%%%


\begin{document}

\pagestyle{plain}




%%%%%%%%%%%%%%%%%%%%%%%%%%%%%%%%%%%% Portada %%%%%%%%%%%%%%%%%%%%%%%%%%%%%%%%%%

%\pagenumbering{gobble}
%\pagenumbering{arabic}

% Universidad, Facultad
\begin{titlepage}
\selectlanguage{spanish}


% logo
\begin{center}
    \includegraphics[scale=0.7]{\logoUniversidad}
\end{center}



\bigskip

\begin{center}
\begin{LARGE}
\escuela \\
\end{LARGE}
\end{center}

\bigskip
\bigskip

% Grado
\begin{center}
\begin{large}
\textbf{\grado}\\
\end{large}
\end{center}

% Curso
\begin{center}
\begin{large}
\textbf{\curso}\\
\end{large}
\end{center}

\bigskip

\textbf{\begin{center}
\begin{large}
\textbf{Trabajo Fin de Grado}
\end{large}
\end{center}}

\bigskip
\bigskip
\bigskip

% Nombre del TFG
\begin{center}
\textbf{\begin{large}
\MakeUppercase{\titulotrabajo}\\
\end{large}}
\end{center}

\begin{center}
    \includegraphics[scale=0.2]{\logoEVS}
\end{center}


% Nombre del autor
\vspace{\fill}
\begin{center}
\textbf{Autor: \nombreautor}\\ \smallskip
% Tutor
\textbf{Tutor: \nombretutor}\\
% Añadir segundo tutor si hubiera


\bigskip

% Fecha
%\textbf{\fecha}\\
\end{center}
\end{titlepage}


%%%%%%%%%%%%%%%%%%%%%%%% Opcional %%%%%%%%%%%%%%%%%%%%%%
%\blankpage

%\thispagestyle{empty}
%\begin{center}

% Nombre del trabajo
%\textbf{\begin{large}
%\MakeUppercase{\titulotrabajo}\\*
%\end{large}}
%\vspace*{0.2cm}
%\vspace{5cm}

% Nombre del autor y del tutor
%\large Autor: \nombreautor \\* \medskip
%\large Tutor: \nombretutor \\*

%\vfill

% Escuela, universidad y fecha
%\escuelalargo \\ \smallskip
%\universidad \\
%\vspace{1cm}
%\fecha \\

%\clearpage

%\end{center}
%%%%%%%%%%%%%%%%%%%%%%%%%%%%%%%%%%%%%%%%%%%%%%%%%%%%%%%%

\hypersetup{pageanchor=true}

\normalsize
\afterpage{\blankpage} % Se deben añadir página en blanco para que lo capítulos de la memoria o estas secciones introductorias empiecen en páginas impares

%%%%%%%%%%%%%%%%%%%%%%%%%%%%%%%%%%%%%%%%%%%%%%%%%%%%%%%%%%%%%%%%%%%%%%%%%%%%%%%





% Estilo de párrafo de los capítulos
\setlength{\parskip}{0.75em}
\renewcommand{\baselinestretch}{1.5}
% Interlineado simple
\spacing{1.5}

\pagenumbering{Roman}
\setcounter{page}{2}


%%%%%%%%%%%%%%%%%%%%%%%%% Agradecimientos o dedicatoria %%%%%%%%%%%%%%%%%%%%%%%%%%%

\chapter*{Agradecimientos}

Breves agradecimientos o dedicatoria.

\afterpage{\blankpage}

%%%%%%%%%%%%%%%%%%%%%%%%%%%%%%%%%%%%%%%%%%%%%%%%%%%%%%%%%%%%%%%%%%%%%%%%%%%%%%%%%%%






%%%%%%%%%%%%%%%%%%%%%%%%%%%%%%%%%%%% Resumen %%%%%%%%%%%%%%%%%%%%%%%%%%%%%%%%%%%%%%

\chapter*{Resumen}

El presente Trabajo de Fin de Grado (TFG) tiene como objetivo principal el desarrollo de una aplicación web llamada Enterprise Event Solutions, 
diseñada para la gestión eficiente de eventos corporativos. Esta herramienta aborda las necesidades específicas de las empresas en la planificación, 
organización y seguimiento de eventos, ofreciendo una solución integral y personalizada.

\section*{Objetivos Principales}
\begin{enumerate}
    \item Facilitar la planificación de eventos: Proveer una plataforma donde las empresas puedan gestionar todos los 
    aspectos relacionados con la organización de eventos.
    \item Reducir la carga administrativa mediante la automatización de tareas repetitivas, como la gestión de inscripciones. 
    \item Permitir a las empresas realizar un seguimiento detallado de los eventos, generando informes y estadísticas que 
    faciliten la toma de decisiones y la evaluación del éxito de los mismos.
    \end{enumerate}

\section*{Resustados Derivados}
\begin{enumerate}
    \item Sistema de gestión de usuarios mediante un usuario Administrador.
    \item Organizaciones con la capacidad de crear, eliminar y editar eventos.
    \item Clientes con la capacidad de navegar entre las diferentes opciones que aporta la interfaz y la adquisición de entradas para estos eventos de empresa.
    \item Se ha desarrollado una interfaz amigable y fácil de usar, que permite a los usuarios navegar y 
    utilizar la plataforma sin necesidad de formación extensa.
    \end{enumerate}

En conclusión, Enterprise Event Solutions se presenta como una solución robusta y eficiente para la gestión de eventos corporativos, 
cumpliendo con los objetivos propuestos y 
demostrando ser una herramienta valiosa para las empresas en la optimización de sus procesos de organización de eventos.

\mbox{} \bigskip

\noindent \textbf{Palabras clave}:
\begin{compactitem}
    \item EVS (Enterprise Event Solutions)
    \item AWS
    \item SpringBoot
    \item Vue
    \item Seguridad
    \item Web
    \item Docker
\end{compactitem}

\afterpage{\blankpage}

%%%%%%%%%%%%%%%%%%%%%%%%%%%%%%%%%%%%%%%%%%%%%%%%%%%%%%%%%%%%%%%%%%%%%%%%%%%%%%%%%%%





%%%%%%%%%%%%%%%%%%%%%%%%%%%%%%%%%%%% Índices %%%%%%%%%%%%%%%%%%%%%%%%%%%%%%%%%%%%

% Estilo de párrafo de los Índices
\setlength{\parskip}{1pt}
\renewcommand{\baselinestretch}{1}
\renewcommand{\contentsname}{Índice de contenidos}


% Índice de contenidos
\tableofcontents
\afterpage{\blankpage}

% Índice de tablas (OPCIONAL)
\listoftables
\afterpage{\blankpage}
\addcontentsline{toc}{chapter}{\noindent \listtablename}

% Índice de figuras (OPCIONAL)
\listoffigures
\afterpage{\blankpage}
\addcontentsline{toc}{chapter}{\listfigurename}

% Índice de códigos/algoritmos (OPCIONAL).   El término "Códigos" se puede cambiar por "Métodos", "Funciones", "Algoritmos", etc.
\renewcommand\lstlistlistingname{Códigos}
\renewcommand\lstlistingname{Código}
\renewcommand\lstlistlistingname{Índice de códigos}

\lstlistoflistings
\afterpage{\blankpage}
\addcontentsline{toc}{chapter}{\lstlistlistingname}


% En este documento (de momento) no se ha considerado incluir un índice de algoritmos/pseudocódigos, como el que aparece en \ref{AdditionalLouvain}

%%%%%%%%%%%%%%%%%%%%%%%%%%%%%%%%%%%%%%%%%%%%%%%%%%%%%%%%%%%%%%%%%%%%%%%%%%%%%%%%%%%





%%%%%%%%%%%%%%%%%%%%%%% Cabeceras y pies de página (Opcional) %%%%%%%%%%%%%%%%%%%%%%%

%\setlength{\headheight}{15.2pt}
\pagestyle{fancy}


\renewcommand{\chaptermark}[1]{\markboth{Capítulo \thechapter.\ #1}{}}

\pagestyle{fancy}
\fancyhf{}
\fancyhead[LO]{\leftmark}
\fancyhead[RO]{}
\fancyhead[RE]{\nouppercase\rightmark}
\fancyhead[LE]{}
\fancyfoot[C]{\thepage}

%%%%%%%%%%%%%%%%%%%%%%%%%%%%%%%%%%%%%%%%%%%%%%%%%%%%%%%%%%%%%%%%%%%%%%%%%%%%%%%%%%%%






%%%%%%%%%%%%%%%%%%%%%%%%%%%%%% Capítulos de la memoria %%%%%%%%%%%%%%%%%%%%%%%%%%%%%



% Capítulo 1
\chapter{Introducción}


%%%%%%%%%%%%%%%%%%%%%%%%%%%%%%%%%%%%%%%%%%%%%%%%%%%%%%%%%%%%%%%%%%%%%%%%%%

% Estilo resto de páginas
\pagestyle{fancy}


% Estilo de párrafo de los capítulos
\setlength{\parskip}{0.75em}
\renewcommand{\baselinestretch}{1.5}
% Interlineado simple
\spacing{1.5}
% Numeración contenido
\pagenumbering{arabic}
\setcounter{page}{1}

%%%%%%%%%%%%%%%%%%%%%%%%%%%%%%%%%%%%%%%%%%%%%%%%%%%%%%%%%%%%%%%%%%%%%%%%%%


\section{Contexto y alcance}

La organización de eventos corporativos es fundamental para fomentar la cultura organizacional, 
establecer relaciones estratégicas e impulsar el crecimiento de una empresa en el mundo moderno. 
Sin embargo, la planificación y ejecución de estos eventos presentan numerosos desafíos que pueden comprometer su éxito. 
Los desafíos comunes incluyen la dificultad de coordinar a varios departamentos, la necesidad de una comunicación fluida y 
la gestión eficiente de los recursos. Debido a esto, surgió la idea de desarrollar Enterprise Event Solutions, una aplicación web destinada a satisfacer estas 
demandas y mejorar significativamente el manejo de eventos corporativos.

La evaluación de una serie de problemas comunes en la gestión de eventos empresariales llevó a la decisión de establecer EVS:

En primer lugar, no hay una plataforma centralizada que permita a las empresas organizar de manera integral todos los aspectos de un evento. 
Muchas empresas dependen de una variedad de programas que no están conectados, como hojas de cálculo, correos electrónicos y software de gestión de 
proyectos, lo que hace que las cosas no funcionen bien y aumenta el riesgo de errores.

La comunicación interna es otro aspecto importante que a menudo se olvida cuando se trata de organizar eventos. La coordinación de equipos y 
departamentos requiere una herramienta que facilite la comunicación en tiempo real y asegure que todos los involucrados estén informados sobre las 
actualizaciones y cambios. El objetivo de EVS es facilitar la comunicación, reducir las confusiones y fomentar la colaboración.

La responsabilidad administrativa asociada con la gestión de eventos. 
La gestión de inscripciones y el seguimiento de 
la asistencia requieren mucho tiempo y recursos. EVS permite a los organizadores 
concentrarse en aspectos más estratégicos y creativos del evento, mejorando su calidad y 
efectividad mediante la automatización de estos procesos.

Contar con herramientas de análisis y seguimiento también es esencial para evaluar el éxito de los eventos y tomar 
decisiones informadas para futuras planificaciones. Las funcionalidades avanzadas de EES permiten la creación de informes 
detallados que brindan a las empresas información útil sobre la participación, el desempeño y las áreas de mejora.

Finalmente, un factor clave en la creación de EVS fue la creciente demanda de soluciones tecnológicas que se adapten a las 
necesidades específicas de cada empresa. La personalización y la flexibilidad de la plataforma la hacen una herramienta adaptable a 
una variedad de tipos de eventos y tamaños de negocios, lo que permite que cada usuario maximice su utilidad.

En resumen, la creación de soluciones de eventos empresariales satisface la necesidad de una herramienta completa y efectiva para la 
gestión de eventos corporativos. El objetivo de EVS es transformar la forma en que las empresas organizan y ejecutan sus eventos, agregando valor y 
contribuyendo al éxito empresarial al centralizar la planificación, mejorar la comunicación, automatizar procesos administrativos y proporcionar 
herramientas de análisis.





% \afterpage{\blankpage} % puede generar problema en índice de contenidos
% \newpage


% Capítulo 2
\chapter{Objetivos}


\section{Objetivos generales}

Para la aplicación de EVS se han marcado las siguientes metas principales:
\begin{itemize}
    \item \textbf{Gestión de la frecuencia de usuarios y eventos en la aplicación.} La aplicación permite gestionar a todos los usuarios en el Sistema
    mediante un usuario Administrador. Esto facilitará la relacion Organización-Cliente mantieniendo un equilibrio entre la cantidad de unos y de otros.
    \item \textbf{Centralizar la planificación de eventos.} La plataforma permite a los usuarios crear y gestionar eventos desde una sola pantalla.
    \item \textbf{Agrupar las Entradas de los usuarios.}  EVS ofrece a los usuarios la posibilidad de Agrupar las entradas en un mismo sitio, así como imprimir las entradas.
    \item \textbf{Aplicación de uso general.} Ofrecer una herramienta para, desde el punto de vista de una organización, mejorar la asistencia a sus eventos
    y, desde el punto de vista del cliente, tener un acceso sencillo e intuitivo a estos.
    \item \textbf{Desarrollo Fullstack.} Aplicar y ampliar los conocimientos adquiridos sobre SpringBoot y Vue.
    \item \textbf{Entrega Continua.} Se han automatizado los procesos de Integración y Despliegue para facilitar el desarrollo.
    \end{itemize}
   
    Se puede hacer uso de la aplicación accediendo al siguiente enlace: \textbf{\href{https://18.133.60.104:8443/}{Enterprise Event Solutions}}

\section{Descripcón del Problema}
Segun datos oficiales, en España hay 2.922.920 empresas de las cuales solo  5.531 son grandes empresas \cite{pymes} ¿Qué pasa con las 2.917.389 restantes?.

\textbf{Problemas de Comunicación y Alcance:}

\begin{enumerate}
    \item \textbf{Recursos Limitados}: A diferencia de las grandes corporaciones, las PYMEs generalmente no disponen de los mismos recursos financieros 
    y humanos para invertir en estrategias de marketing y comunicación. Esto limita su capacidad para desarrollar campañas efectivas y sostenibles que
    lleguen a una amplia audiencia.
    
    \item \textbf{Falta de Canales de Comunicación Adecuados}: Las grandes empresas suelen tener acceso a una variedad de canales de comunicación, 
    incluidos medios de comunicación masiva, redes sociales gestionadas por equipos especializados y eventos de gran escala. 
    Las PYMEs, por otro lado, a menudo carecen de estos canales, lo que dificulta su capacidad para llegar a nuevos clientes y 
    mantener una comunicación constante con sus públicos objetivo.
    
    \item \textbf{Menor Visibilidad}: Las grandes empresas tienen la ventaja de una mayor visibilidad de marca, 
    lo que les permite mantenerse en la mente de los consumidores más fácilmente. Las PYMEs luchan por obtener y mantener esta visibilidad, 
    lo que puede resultar en una menor lealtad del cliente y dificultades para captar nuevos mercados.
    
    \item \textbf{Tecnología y Digitalización}: Muchas PYMEs no tienen acceso a las últimas tecnologías y herramientas digitales que pueden 
    facilitar la comunicación y el marketing. La falta de digitalización no solo afecta su eficiencia operativa sino también su capacidad para 
    implementar estrategias de marketing digital que son cruciales en el mercado actual.
    
    \item \textbf{Reducción de Costes}: La gestión de eventos corporativos es una herramienta clave para la promoción y la creación de redes, 
    pero muchas PYMEs no pueden permitirse los costes asociados con la organización de eventos a gran escala. Esto limita su capacidad para interactuar cara a cara con clientes potenciales y fortalecer las relaciones comerciales existentes.
    
    \item \textbf{Competencia Intensa}: En un mercado saturado, las PYMEs compiten no solo con otras pequeñas empresas sino también con grandes 
    corporaciones que tienen una presencia más consolidada. La competencia intensa puede hacer que sea aún más difícil para las PYMEs destacar y 
    atraer la atención del público.
\end{enumerate}


\textbf{Impacto en el Crecimiento y Sostenibilidad:}

Estos problemas de comunicación y alcance no solo limitan la capacidad de las PYMEs para crecer, sino que también ponen en riesgo su sostenibilidad 
a largo plazo. Sin los medios adecuados para llegar a su público objetivo y sin una estrategia de comunicación efectiva, estas empresas enfrentan 
dificultades para expandirse, innovar y mantenerse competitivas en el mercado. La falta de visibilidad y la incapacidad para interactuar eficazmente 
con los clientes pueden conducir a una disminución de las ventas y, en última instancia, afectar la viabilidad económica de la empresa.

\textbf{Necesidad de Soluciones Eficientes:}

Para abordar estos desafíos, surge la necesidad de soluciones eficientes y accesibles que puedan ayudar a las PYMEs a mejorar su comunicación y alcance. 
Herramientas que centralicen la gestión de eventos, faciliten la automatización de procesos administrativos, mejoren la visibilidad y proporcionen canales 
de comunicación efectivos son esenciales para permitir que estas empresas compitan en igualdad de condiciones con las grandes corporaciones. 
\textbf{Enterprise Event Solutions} se presenta como una respuesta a estas necesidades, ofreciendo una plataforma integral diseñada específicamente 
para ayudar a las PYMEs a superar sus limitaciones y alcanzar sus objetivos de crecimiento y sostenibilidad.


\section{Estudio de Alternativas}
Contar con las herramientas adecuadas en el campo de la organización de eventos es esencial para el éxito. Debido a su amplia gama de características, 
las aplicaciones de gestión de eventos se han convertido en una herramienta esencial para simplificar tareas, optimizar procesos y 
permitir la realización de eventos memorables.

A continuación se muestra una serie de Aplicaciones Web con funcionalidades parecidas a Enterprise Event Solutions:

\begin{itemize}
    \item \textbf{Eventbrite} es una plataforma versátil para la creación y gestión de eventos. Permite a los usuarios vender entradas online 
    con opciones de precios y tarifas personalizables, así como gestionar el registro y la participación de los asistentes. Además, 
    ofrece herramientas de marketing integradas para promocionar los eventos y análisis de datos detallados sobre la asistencia y 
    el rendimiento de los mismos.
    \item \textbf{Wix Events} es una herramienta de creación de páginas de 
    eventos personalizadas dentro del sitio web de Wix. Ofrece funcionalidades para gestionar el registro de los asistentes y 
    proporciona herramientas de marketing para promocionar los eventos. Además, permite a los usuarios realizar un análisis 
    detallado de la asistencia y el rendimiento del evento a través de datos recopilados en la plataforma.
    \item \textbf{Zoom Events} es una plataforma diseñada para la creación y gestión de eventos virtuales e híbridos, como webinars, 
    reuniones y conferencias. Permite interactuar con los participantes a través de funciones como encuestas, preguntas y respuestas, 
    y chat en vivo. Además, ofrece integraciones con otras herramientas de Zoom para una experiencia más completa y proporciona análisis 
    de datos sobre la asistencia y el rendimiento del evento.
\end{itemize}

El valor añadido distintivo de \textbf{Enterprise Event Solutions}, reside en el enfoque centrado en el usuario y la excepcional 
facilidad de uso. Se ha diseñado la plataforma desde sus cimientos con la premisa de que la accesibilidad y la usabilidad son fundamentales 
para garantizar una experiencia óptima para todos los usuarios, independientemente de su nivel de familiaridad con la tecnología.

Lo que distingue a \textbf{Enterprise Event Solutions} es la capacidad para llegar a usuarios de todos los niveles de habilidad tecnológica. 
Reconozco que no todos los usuarios están familiarizados con las últimas tecnologías, y es por eso que he priorizado la accesibilidad en el 
diseño de mi plataforma. Incluso aquellos que pueden sentirse desactualizados en términos de tecnología encontrarán que \textbf{EVS} es fácil 
de entender y utilizar, gracias a una interfaz intuitiva que les ayuda a navegar por todas las funcionalidades sin dificultad.

\section{Metodologías Empleadas}

En el desarrollo de Enterprise Event Solutions, se ha adoptado una metodología ágil basada en Scrum para asegurar que el proceso de creación 
fuera eficiente, organizado y adaptable a cambios y mejoras constantes. Aunque he trabajado solo en este proyecto, se ha implementado de manera 
rigurosa los principios de Scrum, enfocanose en alcanzar pequeños objetivos semanales y manteniendo una estructura organizada.

Cada semana, se han establecido objetivos específicos y alcanzables, lo que ha permitido avanzar de manera constante y mantener un alto nivel 
de motivación. Esta división en pequeños objetivos semanales ha ayudado a gestionar mejor el tiempo y a priorizar tareas, asegurando que cada 
funcionalidad de la aplicación se desarrolle de manera coherente y sin omisiones.

Para organizar y seguir el progreso, se ha utilizado un tablero de Trello. Este tablero ha sido una herramienta invaluable para no olvidar ideas o 
tareas pendientes. Cada tarjeta en el tablero representaba una tarea o idea específica, y he seguido un flujo de trabajo que incluía las siguientes 
etapas: por hacer, en progreso, y completado. Esta visualización clara del trabajo pendiente y del progreso realizado me ha permitido mantenerme 
enfocado y productivo.

Además, al final de cada semana, he realizado una revisión de los objetivos alcanzados y he ajustado el plan para la semana siguiente. Esta práctica 
de retrospectiva semanal ha sido fundamental para identificar áreas de mejora, solucionar problemas y asegurar que el proyecto siga avanzando de acuerdo 
con los objetivos establecidos.

En resumen, la adopción de la metodología Scrum, con su enfoque en pequeños objetivos semanales y el uso de un tablero Trello para la gestión de 
tareas, ha sido clave para el desarrollo exitoso de Enterprise Event Solutions. A pesar de trabajar solo, esta estructura me ha permitido mantener 
un alto nivel de organización, adaptabilidad y eficiencia en todo el proceso de desarrollo.
\newpage

\begin{figure}[h]
    \includegraphics[width=\linewidth]{Trello.png}
    \caption{Trello en proceso.}
    \label{fig:metodologias1}
\end{figure}

En cuanto al desarrollo del código, se ha utilizado un flujo de trabajo basado en Pull Requests (PRs). Cada cambio o nueva 
funcionalidad se desarrollaba en una rama separada y se integraba al código principal solo después de ser revisada y aprobada 
mediante un Pull Request. Este enfoque ha permitido mantener un historial claro de los cambios, realizar revisiones detalladas y 
asegurar que cada nueva adición al código se integrara de manera ordenada y sin conflictos. Este método ha ayudado a mantener la disciplina y la organización en el
 desarrollo del software.

El uso de Pull Requests ha ofrecido múltiples ventajas en el proceso de desarrollo. En primer lugar, ha facilitado la identificación y 
resolución de errores antes de que los cambios se integren en la rama principal. Cada PR actuaba como un punto de control donde podía revisar 
el código, realizar pruebas y verificar la funcionalidad, asegurando que solo los cambios bien testeados y verificados fueran añadidos al proyecto.

Además, este método ha proporcionado una documentación implícita del desarrollo. Cada Pull Request incluía una descripción detallada de los 
cambios realizados, los problemas que solucionaba o las nuevas características añadidas. Esto no solo facilitó la gestión del proyecto, sino 
que también creó un registro histórico útil para futuras referencias y para cualquier otra persona que pueda colaborar en el futuro.

Trabajar con ramas separadas para cada nueva funcionalidad o cambio también ha sido crucial para mantener la estabilidad del proyecto. Al aislar el 
desarrollo de nuevas características en ramas dedicadas, he evitado que los cambios en progreso afecten la estabilidad de la rama principal. Esto ha 
sido especialmente útil para realizar experimentos o implementar grandes cambios sin riesgo de romper la aplicación.

Finalmente, la disciplina de utilizar Pull Requests, incluso trabajando solo, ha fomentado un enfoque metódico y estructurado al desarrollo. Este 
flujo de trabajo me ha obligado a pensar críticamente sobre cada cambio, a documentarlo adecuadamente y a asegurarse de que cada PR cumpliera con los 
estándares de calidad antes de ser fusionado. De hecho, para asegurar que todo cambio que se realizara estuviera controlado por si fuera necesario volver a versiones anteriores
hasta los cambios más pequeños se realizaban mediante PRs. Gracias a esto he consegido revertir cambios en momentos críticos del desarrollo.
\newpage
\begin{figure}[h]
    \centering
    \includegraphics[width=\linewidth]{PRs.png}
    \caption{PRs del Repositorio.}
    \label{fig:metodologias2}
\end{figure}

En resumen, la adopción de la metodología Scrum, con su enfoque en pequeños objetivos semanales y el uso de un tablero Trello para la gestión de tareas, 
junto con el uso de Pull Requests, ha sido clave para el desarrollo exitoso de Enterprise Event Solutions.

El total desarrollo se ha realizado en un repositorio de GitHub donde se ha incluido un Readme.md utíl para el tanto el despliegue en producción como en desarrollo. De esta forma
aún sin la totalidad de los requerimientos, ya sea una cuenta de AWS o de Gmail, se podría hacer uso de la aplicación de forma local.


\blankpage

% Capítulo 3
\chapter{Descripción Informática}
\label{chap:contenidos}

En esta sección se entrará a describir con total detalle el contenido técnico de la aplicación. Partiendo desde un analisis de requisitos funcionales
hasta no funcionales, pasando por las tecnologías usadas para la implemetación de estos. 

\section{Tecnologias Empleadas}

\subsection{SpringBoot}
Spring Boot es un marco de trabajo Java que simplifica el desarrollo de aplicaciones web al eliminar la 
complejidad de la configuración manual. Ofrece configuración automática, empaquetado de aplicaciones independientes y un inicio rápido integrado, 
lo que permite a los desarrolladores crear aplicaciones de manera más rápida y eficiente, centrándose en la lógica de negocio en lugar de la configuración. 
Genial para el desarrollo de aplicaciones \textit{CRUD}\footnote{\textit{CRUD} es un acrónimo en inglés que se refiere a las operaciones 
básicas de manipulación de datos en aplicaciones informáticas: Create (Crear), Read (Leer), Update (Actualizar) y Delete (Eliminar).}.
\subsection{Maven}
Maven es una herramienta de gestión de proyectos Java que simplifica la estructuración, gestión de dependencias y automatización de la construcción. 
Permite definir dependencias en un archivo de configuración (pom.xml), ejecutar fases de construcción estandarizadas y gestionar repositorios 
centralizados de bibliotecas. Es ampliamente utilizado en el desarrollo Java por su integración con IDEs y su capacidad para proyectos multiproyecto, es decir,
permite la gestión de múltiples proyectos en un solo repositorio, lo que facilita la colaboración entre desarrolladores y la gestión.
\subsection{Vue.js}
Vue.js es un popular marco de trabajo de código abierto para construir interfaces de usuario interactivas en aplicaciones web de una sola página. 
Se destaca por su enfoque centrado en el componente, su sintaxis declarativa y su eficiente sistema de reactividad. Vue.js facilita la construcción 
de aplicaciones web complejas al fomentar la reutilización de componentes y ofrecer herramientas integradas para la manipulación del DOM y la gestión 
del estado de la aplicación.

\subsection{MySQL}
MySQL es un sistema de gestión de bases de datos relacional de código abierto ampliamente utilizado en el desarrollo de 
aplicaciones web y empresariales. Destaca por su escalabilidad, rendimiento, fiabilidad y facilidad de uso. Ofrece opciones 
sólidas de seguridad y es compatible con una variedad de plataformas y lenguajes de programación. MySQL es una herramienta poderosa 
para gestionar eficientemente grandes volúmenes de datos y garantizar la integridad y disponibilidad de la información.

\subsection{IntelliJ}
IntelliJ IDEA es un entorno de desarrollo integrado (IDE) altamente productivo desarrollado por JetBrains. 
Diseñado para diversas tecnologías, ofrece una interfaz de usuario intuitiva y funciones avanzadas que mejoran la 
productividad del desarrollador. Con soporte para múltiples lenguajes y marcos de trabajo, integración con herramientas de desarrollo y 
características colaborativas, IntelliJ IDEA es una opción popular para el desarrollo de software en Java y otros lenguajes.

\subsection{VSCode}
Visual Studio Code (VSCode) es un IDE popular desarrollado por Microsoft, conocido por su versatilidad, rendimiento y comunidad activa.
 Ofrece características avanzadas y extensiones para diferentes lenguajes de programación. Es altamente personalizable y cuenta con una 
 amplia documentación disponible en su sitio web oficial.

\subsection{AWS S3}
AWS S3 es un servicio de almacenamiento en la nube ofrecido por Amazon Web Services. Destaca por su escalabilidad, durabilidad, seguridad y 
facilidad de uso. Permite almacenar grandes cantidades de datos de forma segura y acceder a ellos de manera eficiente a través de una interfaz 
intuitiva y una API robusta. Con una estructura de precios flexible, AWS S3 es una opción atractiva para empresas que buscan una solución de 
almacenamiento en la nube rentable y confiable.

\subsection{AWS EC2}
AWS EC2 es un servicio de cómputo en la nube proporcionado por Amazon Web Services. Destaca por su escalabilidad, flexibilidad y seguridad. 
Permite a los usuarios alquilar capacidad informática según sus necesidades, con opciones de pago por uso. Es fácil de usar y ofrece una variedad 
de opciones de implementación. En resumen, EC2 es una solución eficiente y confiable para ejecutar aplicaciones y cargas de trabajo en la nube.

\subsection{Docker}
Docker es una plataforma de código abierto que permite crear, implementar y ejecutar aplicaciones en contenedores. Destaca por su portabilidad, 
eficiencia y facilidad de uso. Con la tecnología de contenedores, Docker ofrece aislamiento de aplicaciones y escalabilidad, lo que simplifica el 
desarrollo y la administración de aplicaciones en diferentes entornos. En resumen, Docker es una herramienta fundamental para la creación y gestión 
eficiente de aplicaciones modernas.

\newpage
\section{Resumen de las tecnologias}
La Tabla \ref{tabla:tecnologias_usos} recoge un resumen de las tecnologías empleadas para el desarrollo de Enterprise Event Solutions
\begin{table}[h]
\begin{tabular}{ p{3cm} l  }

    \hline
    Tecnologia& Uso \\
    \hline
    SpringBoot   & Framework \\
    Maven &   Framework \\
    Vue.js & Framework  \\
    MySQL    & Tecnologia de BD \\
    IntelliJ&   Entorno de Desarrollo  \\
    VSCode& Entorno de Desarrollo \\
    AWS S3& Tecnologia de de BD en la nube  \\
    AWS EC2& Servidor web en la nube  \\
    Docker& Crecion de imagenes comprimidas para despliegue  \\
    \hline
   \end{tabular}
   \caption{Tecnologías y sus usos correspondientes}
   \label{tabla:tecnologias_usos}
\end{table}
\newpage
\section{Arquitectura}
\textbf{Enterprise Event Solutions} es una aplicación \textbf{Fullstack} creada siguiendo un patrón de diseño MVC. A continuación, se muestra un plano general
que será desglosado:  CAMBIAR DIAGRAMA !!!!1 AÑADIR SEGURIDAD 

\section*{Arquitectura General}
\begin{figure}[h]
    \centering
    \includegraphics[width=0.8\textwidth]{Arquitectura.png} 
    \caption{Diagrama de Arquitectura de EVS}
    \label{fig:mvc_architecture}
\end{figure}
\newpage

\section*{Backend (Modelo y Controlador)}

\begin{itemize}
    \item \textbf{Tecnología}: \textbf{SpringBoot}
    \item \textbf{Descripción}: Utilizado por su gran versatilidad para manejar la lógica de negocio y control de la aplicación. Gracias a esto, el frontend
    de la aplicación se sirve directamente desde el Servidor de Apache proporcionado por spring, es decir, tengo empaquetado el tanto el back como el front en la misma ip
    y en el mismo puerto.
\end{itemize}

\section*{Frontend (Vista)}

\begin{itemize}
    \item \textbf{Tecnología}: \textbf{Vue.js}
    \item \textbf{Descripción}: Seleccionado para explorar otras tecnologías y facilitar la creación de una interfaz de usuario interactiva.
\end{itemize}

\section*{Tecnologías Complementarias}

\begin{itemize}
    \item \textbf{AWS EC2}: 
    \begin{itemize}
        \item \textbf{Descripción}: Instancia EC2 para desplegar la aplicación, ofreciendo un entorno robusto y escalable.
    \end{itemize}
    \item \textbf{AWS S3}:
    \begin{itemize}
        \item \textbf{Descripción}: Bucket S3 para almacenar imágenes, reduciendo la carga de trabajo en la base de datos.
    \end{itemize}
    \item \textbf{MySQL}:
    \begin{itemize}
        \item \textbf{Descripción}: Base de datos utilizada para gestionar los datos de la aplicación de manera eficiente.
    \end{itemize}
\end{itemize}

\textbf{Enterprise Event Solutions} combina las fortalezas de \textbf{SpringBoot} y \textbf{Vue.js} en un patrón MVC para ofrecer una solución robusta y 
escalable. Con el soporte de \textbf{AWS} y \textbf{MySQL}, la aplicación está preparada para manejar grandes volúmenes de datos y ofrecer una experiencia 
de usuario fluida.


Cada uno de los componentes definidos en la Imagen \ref{fig:mvc_architecture} será desglosado en las próximas secciones para explicar detalladamente el contenido de 
la aplicación.

\subsection{Modelo}
La persistencia en Enterprise Event Solutions se sustenta en una serie de Entidades almacenadas en una base de datos con la que los servicios interactuan mediante
las interfaces proporcionadas por JPA \ref{sec:jpa}. Es el pilar fundamental en el que se sustenta el modelo de negocio de la aplicación.
\begin{figure}[h]
    \centering
    \includegraphics[width=1.2\textwidth]{EVSdiagra.png} 
    \caption{Modelo E-R de la BD}
    \label{fig:diagramaBD}
\end{figure}

\subsubsection{JPA}
\label{sec:jpa}

JPA (Java Persistence API) es una especificación de Java que facilita la gestión de la persistencia de datos en aplicaciones Java.
En Spring, se usa para simplificar las operaciones CRUD y el manejo de relaciones entre entidades, permitiendo trabajar con objetos en lugar de SQL. 
Spring Data JPA integra JPA en el ecosistema Spring, proporcionando repositorios predefinidos para operaciones de base de datos comunes.

A continuación se muestra un fragmento de código de Spring Data JPA en Java:
\myjavastyle
\begin{lstlisting}[language=Java, caption=Ejemplo de Repositorio en Spring Data JPA]
@Repository
public interface UserRepository extends JpaRepository<User,Long> {
    public Optional<User> findByEmail(String email);
    public List<User> findAllByRole(UserTipeEnum type);
    public Optional<User> findByUsername(String username);
}
\end{lstlisting}

Como se puede observar, gracias a la gran potencia de Spring y de JPA, simplemente con crear una interfaz con metodos que creemos que vamos a necesitar en 
nuestros servicios, y sin la necesidad de añadir @Querys, podemos hacer consultas sobre la base de datos.

Pero también podemos hacer nuestras propias Querys sobre la base de datos:
\myjavastyle
\begin{lstlisting}[language=Java, caption=Ejemplo de Query  en Spring Data JPA]
    @Transactional
    @Modifying
    @Query("UPDATE Event e SET e.current_capacity = e.current_capacity + 1 WHERE e.id = :id AND e.current_capacity + 1  <= e.max_capacity")
    public int incrementCurrentCapacity(@Param("id") Long id);
\end{lstlisting}

\subsection{Contolador}
La creación de controladores me ha permitido gestionar un sistema de endpoints que sirve como la interfaz principal 
para la comunicación entre el cliente y el servidor. Estos controladores se encargan de recibir las solicitudes HTTPS de delegar el procesamiento a 
los servicios correspondientes. Los servicios encapsulan la lógica de negocio y se comunican con los repositorios JPA para acceder a los datos de la 
base de datos de manera eficiente. Además, para garantizar la seguridad, se utiliza JWT (JSON Web Tokens), que permite la autenticación y autorización 
de los usuarios. Los tokens JWT se generan tras un inicio de sesión exitoso y se utilizan para proteger los endpoints, asegurando que solo los usuarios 
autenticados puedan acceder a recursos específicos. Este enfoque no solo organiza el código de manera modular y mantenible, sino que también proporciona 
una robusta capa de seguridad para la API.
\begin{figure}[h]
    \centering
    \includegraphics[width=1\textwidth]{DiagramaClases.png} 
    \caption{Diagrama de Clases de EVS}
    \label{fig:class_architecture}
\end{figure}

Además he implementado otras configuraciones para añadir una capa extra de seguridad a la aplicación como configuración CSRFH y Cors para limitar las url 
que pueden hacer uso de los endpoints de mi app. Por supuesto he creado las configuraciones necesarias para manejar las tecnologias externas como S3, dentro de 
EVS.
\begin{figure}[h]
    \centering
    \includegraphics[width=1\textwidth]{security.png} 
    \caption{Seguridad del backend}
    \label{fig:securityClasses}
\end{figure}

\subsection{Vista}
El frontend de Enterprise Event Solution ha sido implementado en Vue.js. Especificamente en Vue3, que trae consigo algunos cambios con respecto a su versión
anterior Vue2. La utilización de este framework me ha permitido añadir tanto bibliotecas de componentes y gráficos (Bootstrap5 y Chart.js respectivamente)
como relaciones entre los diferentes componentes basado en "props", permitiendo de esta manera diferentes comportamientos del componente dependiendo de
que valores se le pasen en el componente padre.

Un ejemplo del uso de estas props sería:
\myvuestyle
\begin{lstlisting}[language=HTML, caption=Ejemplo del Padre, label=lst:padre]
    <event_cards
    :evento="evento"
    :is-org="false"
    >
    </event_cards>
\end{lstlisting}
\myvuestyle
\begin{lstlisting}[language=HTML, caption=Ejemplo del Hijo, label=lst:hijo]
    <script lang="ts">
    import {EventService} from "../services/event.service";
    import { useRouter } from "vue-router";
    import { Event } from "../models/Event";
    import { computed, ref } from "vue";
    export  default {
        name: "event_cards",
        props:{
          evento: Object as ()=> Event,
          isOrg: Boolean,
        },
    }
\end{lstlisting}

En el código \ref{lst:padre}, se observa que se inserta en el template un componente con la información del evento. Por otro lado, en el código 
\ref{lst:hijo}, se manejan estos valores que hemos pasado desde el componente padre. En este fragmento, si se le pasa \texttt{true} en la variable 
\texttt{is-org}, el componente tendrá un comportamiento diferente a si se le pasa \texttt{false}. Ademas en la variable \texttt{evento} le pasamos la información
del evento para garantizar la reactividad de los componetes de forma individual. De esta forma trabajamos con módulos y no con un elemento.

\blankpage

% Nuevo capítulo

\chapter{Conclusiones y trabajos futuros}


\section{Futuros proyectos}
Como creador de Enterprise Event Solutions, y el significado que tiene para mí, tengo intención de continuar mejorando Enterprise Event Solutions hasta
su límite. Algunas funcionalidades extra que podría incluir serían:
\begin{itemize}
    \item Software de lectura de QR integrada: Validar las entradas desde la propia aplicación para poder acceder al evento.
    \item Pasarela de pago: Los eventos que sean de pago puedan ser abonados desde la propia aplicación. 
    \item Ampliar el sistema de Correo: Añadir correos de alerta cuando se aproxima la fecha de un evento.
\end{itemize}

Las posibilidades con una aplicación de este tipo son infinitas. Estas son algunas de las posibles mejoras a corto-medio plazo. 

\section{Conclusiones}
Pese a que ya existen numerosas aplicaciones para gestionar eventos, he querido darle otro enfoque centrado en el la gente de a pie y en las pequeñas empresas,
EVS permite concentrar la gestión de clientes y eventos en un mismo sitio ahorrando infinidad de recursos tanto materiales como personales. 

La gran aceptación que ha tenido entre los participantes del Estudio de Caso es para mí una recompensa al esfuerzo y las horas dedicadas a este TFG, y por supuesto
me animan a continuar con el desarrollo.

Por otro lado me gustaría recalcar que, gracias a realizar este proyecto, he adquirido conocimientos útiles para el desarrollo laboral y personal y por 
supuesto no se quedará ahí ya que he descubierto que cada vez me gusta más el mundo del Desarrollo de Aplicaciones Web aunque si me tengo que decantar, me quedo con
el Backend.




\blankpage


%%%%%%%%%%%%%%%%%%%%%%%%%%%%%%% Bibliografía %%%%%%%%%%%%%%%%%%%%%%%%%%%%%%%

\phantomsection
\addcontentsline{toc}{chapter}{Bibliografía}

\footnotesize{
%\bibliographystyle{hispa}
\bibliographystyle{IEEEtran}
\bibliography{bibliografia}
}



% No expandir elementos para llenar toda la página
\raggedbottom
\afterpage{\blankpage}

\newpage




%%%%%%%%%%%%%%%%%%%%%%%%%%%%%%% Apéndices %%%%%%%%%%%%%%%%%%%%%%%%%%%%%%%

\appendix

\phantomsection
\addcontentsline{toc}{chapter}{Apéndices}

\mbox{}
\vfill
\begin{center}
\begin{Huge}
\textbf{Apéndices}
\end{Huge}
\end{center}
\vfill
\mbox{}
\thispagestyle{empty}

\newpage
\mbox{}
\thispagestyle{empty}
\newpage


% Primer apéndice
\chapter{Este es el primer apéndice}
\label{sec:apendice}

\section{Formulario Estudio de Caso}

A continuación se muestran los datos recogidos del Estudio de Caso: \textbf{\href{https://forms.gle/vdjaMq6vTJrPTLeh7}{Enlace a Formulario EVS}}

\begin{figure}[h]
    \centering
    \includegraphics[width=0.9\textwidth]{Form1.png} 
    \caption{Formulario Pregunta 1}
    \label{fig:form1}
\end{figure}
\begin{figure}[h]
    \centering
    \includegraphics[width=0.9\textwidth]{Form2.png} 
    \caption{Formulario Pregunta 2}
    \label{fig:form2}
\end{figure} 
\begin{figure}[h]
    \centering
    \includegraphics[width=0.9\textwidth]{Form3.png} 
    \caption{Formulario Pregunta 3}
    \label{fig:form3}
\end{figure}
\begin{figure}[h]
    \centering
    \includegraphics[width=0.9\textwidth]{Form4.png} 
    \caption{Formulario Pregunta 4}
    \label{fig:form4}
\end{figure}
\begin{figure}[h]
    \centering
    \includegraphics[width=0.9\textwidth]{Form5.png} 
    \caption{Formulario Pregunta 5}
    \label{fig:form5}
\end{figure}
\begin{figure}[h]
    \centering
    \includegraphics[width=0.9\textwidth]{Form6.png} 
    \caption{Formulario Pregunta 6}
    \label{fig:form6}
\end{figure}
\begin{figure}[h]
    \centering
    \includegraphics[width=0.9\textwidth]{Form7.png} 
    \caption{Formulario Pregunta 7}
    \label{fig:form7}
\end{figure}
\begin{figure}[h]
    \centering
    \includegraphics[width=0.9\textwidth]{Form8.png} 
    \caption{Formulario Pregunta 8}
    \label{fig:form8}
\end{figure}
\begin{figure}[h]
    \centering
    \includegraphics[width=0.9\textwidth]{Form9.png} 
    \caption{Formulario Pregunta 9}
    \label{fig:form9}
\end{figure}


% Fin del documento
\end{document}
