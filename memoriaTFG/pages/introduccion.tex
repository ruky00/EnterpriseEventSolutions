
\section{Contexto y alcance}

La organización de eventos corporativos es fundamental para fomentar la cultura organizacional, 
establecer relaciones estratégicas e impulsar el crecimiento de una empresa en el mundo moderno. 
Sin embargo, la planificación y ejecución de estos eventos presentan numerosos desafíos que pueden comprometer su éxito. 
Los desafíos comunes incluyen la dificultad de coordinar a varios departamentos, la necesidad de una comunicación fluida y 
la gestión eficiente de los recursos. Debido a esto, surgió la idea de desarrollar Enterprise Event Solutions, una aplicación web destinada a satisfacer estas 
demandas y mejorar significativamente el manejo de eventos corporativos.

La evaluación de una serie de problemas comunes en la gestión de eventos empresariales llevó a la decisión de establecer EVS:

En primer lugar, no hay una plataforma centralizada que permita a las empresas organizar de manera integral todos los aspectos de un evento. 
Muchas empresas dependen de una variedad de programas que no están conectados, como hojas de cálculo, correos electrónicos y software de gestión de 
proyectos, lo que hace que las cosas no funcionen bien y aumenta el riesgo de errores.

La comunicación interna es otro aspecto importante que a menudo se olvida cuando se trata de organizar eventos. La coordinación de equipos y 
departamentos requiere una herramienta que facilite la comunicación en tiempo real y asegure que todos los involucrados estén informados sobre las 
actualizaciones y cambios. El objetivo de EVS es facilitar la comunicación, reducir las confusiones y fomentar la colaboración.

La responsabilidad administrativa asociada con la gestión de eventos. 
La gestión de inscripciones y el seguimiento de 
la asistencia requieren mucho tiempo y recursos. EVS permite a los organizadores 
concentrarse en aspectos más estratégicos y creativos del evento, mejorando su calidad y 
efectividad mediante la automatización de estos procesos.

Contar con herramientas de análisis y seguimiento también es esencial para evaluar el éxito de los eventos y tomar 
decisiones informadas para futuras planificaciones. Las funcionalidades avanzadas de EVS permiten la creación de informes 
detallados que brindan a las empresas información útil sobre la participación, el desempeño y las áreas de mejora.

Finalmente, un factor clave en la creación de EVS fue la creciente demanda de soluciones tecnológicas que se adapten a las 
necesidades específicas de cada empresa. La personalización y la flexibilidad de la plataforma la hacen una herramienta adaptable a 
una variedad de tipos de eventos y tamaños de negocios, lo que permite que cada usuario maximice su utilidad.

En resumen, la creación de soluciones de eventos empresariales satisface la necesidad de una herramienta completa y efectiva para la 
gestión de eventos corporativos. El objetivo de EVS es transformar la forma en que las empresas organizan y ejecutan sus eventos, agregando valor y 
contribuyendo al éxito empresarial al centralizar la planificación, mejorar la comunicación, automatizar procesos administrativos y proporcionar 
herramientas de análisis.



