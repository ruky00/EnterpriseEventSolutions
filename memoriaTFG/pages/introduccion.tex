
\section{Contexto y alcance}

La organización de eventos corporativos es fundamental para fomentar la cultura organizacional, 
establecer relaciones estratégicas e impulsar el crecimiento de una empresa en el mundo moderno. 
Sin embargo, la planificación y ejecución de estos eventos presentan numerosos desafíos que pueden comprometer su éxito. 
Los desafíos comunes incluyen la dificultad de coordinar a varios departamentos, la necesidad de una comunicación fluida y 
la gestión eficiente de los recursos. Debido a esto, surgió la idea de desarrollar Enterprise Event Solutions (EVS), una aplicación web destinada a satisfacer estas 
demandas y mejorar significativamente el manejo de eventos corporativos.

La evaluación de una serie de problemas comunes en la gestión de eventos empresariales llevó a la decisión de establecer EVS:

En primer lugar, no hay una plataforma centralizada que permita a las empresas organizar de manera integral todos los aspectos de un evento. 
Muchas empresas dependen de una variedad de programas que no están conectados, como hojas de cálculo, correos electrónicos y software de gestión de 
proyectos, lo que hace que las cosas no funcionen bien y aumenta el riesgo de errores.

La comunicación interna es otro aspecto importante que a menudo se olvida cuando se trata de organizar eventos. La coordinación de equipos y 
departamentos requiere una herramienta que facilite la comunicación en tiempo real y asegure que todos los involucrados estén informados sobre las 
actualizaciones y cambios. El objetivo de EVS es facilitar la comunicación, reducir las confusiones y fomentar la colaboración.

La responsabilidad administrativa asociada con la gestión de eventos. 
La gestión de inscripciones y el seguimiento de 
la asistencia requieren mucho tiempo y recursos. EVS permite a los organizadores 
concentrarse en aspectos más estratégicos y creativos del evento, mejorando su calidad y 
efectividad mediante la automatización de estos procesos.

Contar con herramientas de análisis y seguimiento también es esencial para evaluar el éxito de los eventos y tomar 
decisiones informadas para futuras planificaciones. Las funcionalidades avanzadas de EVS permiten la creación de informes 
detallados que brindan a las empresas información útil sobre la participación, el desempeño y las áreas de mejora.

Finalmente, un factor clave en la creación de EVS fue la creciente demanda de soluciones tecnológicas que se adapten a las 
necesidades específicas de cada empresa. La personalización y la flexibilidad de la plataforma la hacen una herramienta adaptable a 
una variedad de tipos de eventos y tamaños de negocios, lo que permite que cada usuario maximice su utilidad.

En resumen, la creación de soluciones de eventos empresariales satisface la necesidad de una herramienta completa y efectiva para la 
gestión de eventos corporativos. El objetivo de EVS es transformar la forma en que las empresas organizan y ejecutan sus eventos, agregando valor y 
contribuyendo al éxito empresarial al centralizar la planificación, mejorar la comunicación, automatizar procesos administrativos y proporcionar 
herramientas de análisis.

\section{Descripcón del Problema}
Segun datos oficiales, en España hay 2.922.920 empresas de las cuales solo  5.531 son grandes empresas \cite{pymes} ¿Qué pasa con las 2.917.389 restantes?.

\textbf{Problemas de Comunicación y Alcance:}

\begin{enumerate}
    \item \textbf{Recursos Limitados}: A diferencia de las grandes corporaciones, las PYMEs generalmente no disponen de los mismos recursos financieros 
    y humanos para invertir en estrategias de marketing y comunicación. Esto limita su capacidad para desarrollar campañas efectivas y sostenibles que
    lleguen a una amplia audiencia.
    
    \item \textbf{Falta de Canales de Comunicación Adecuados}: Las grandes empresas suelen tener acceso a una variedad de canales de comunicación, 
    incluidos medios de comunicación masiva, redes sociales gestionadas por equipos especializados y eventos de gran escala. 
    Las PYMEs, por otro lado, a menudo carecen de estos canales, lo que dificulta su capacidad para llegar a nuevos clientes y 
    mantener una comunicación constante con sus públicos objetivo.
    
    \item \textbf{Menor Visibilidad}: Las grandes empresas tienen la ventaja de una mayor visibilidad de marca, 
    lo que les permite mantenerse en la mente de los consumidores más fácilmente. Las PYMEs luchan por obtener y mantener esta visibilidad, 
    lo que puede resultar en una menor lealtad del cliente y dificultades para captar nuevos mercados.
    
    \item \textbf{Tecnología y Digitalización}: Muchas PYMEs no tienen acceso a las últimas tecnologías y herramientas digitales que pueden 
    facilitar la comunicación y el marketing. La falta de digitalización no solo afecta su eficiencia operativa sino también su capacidad para 
    implementar estrategias de marketing digital que son cruciales en el mercado actual.
    
    \item \textbf{Reducción de Costes}: La gestión de eventos corporativos es una herramienta clave para la promoción y la creación de redes, 
    pero muchas PYMEs no pueden permitirse los costes asociados con la organización de eventos a gran escala. Esto limita su capacidad para interactuar cara a cara con clientes potenciales y fortalecer las relaciones comerciales existentes.
    
    \item \textbf{Competencia Intensa}: En un mercado saturado, las PYMEs compiten no solo con otras pequeñas empresas sino también con grandes 
    corporaciones que tienen una presencia más consolidada. La competencia intensa puede hacer que sea aún más difícil para las PYMEs destacar y 
    atraer la atención del público.
\end{enumerate}


\textbf{Impacto en el Crecimiento y Sostenibilidad:}

Estos problemas de comunicación y alcance no solo limitan la capacidad de las PYMEs para crecer, sino que también ponen en riesgo su sostenibilidad 
a largo plazo. Sin los medios adecuados para llegar a su público objetivo y sin una estrategia de comunicación efectiva, estas empresas enfrentan 
dificultades para expandirse, innovar y mantenerse competitivas en el mercado. La falta de visibilidad y la incapacidad para interactuar eficazmente 
con los clientes pueden conducir a una disminución de las ventas y, en última instancia, afectar la viabilidad económica de la empresa.

\textbf{Necesidad de Soluciones Eficientes:}

Para abordar estos desafíos, surge la necesidad de soluciones eficientes y accesibles que puedan ayudar a las PYMEs a mejorar su comunicación y alcance. 
Herramientas que centralicen la gestión de eventos, faciliten la automatización de procesos administrativos, mejoren la visibilidad y proporcionen canales 
de comunicación efectivos son esenciales para permitir que estas empresas compitan en igualdad de condiciones con las grandes corporaciones. 
\textbf{Enterprise Event Solutions} se presenta como una respuesta a estas necesidades, ofreciendo una plataforma integral diseñada específicamente 
para ayudar a las PYMEs a superar sus limitaciones y alcanzar sus objetivos de crecimiento y sostenibilidad.


\section{Estudio de Alternativas}
Contar con las herramientas adecuadas en el campo de la organización de eventos es esencial para el éxito. Debido a su amplia gama de características, 
las aplicaciones de gestión de eventos se han convertido en una herramienta esencial para simplificar tareas, optimizar procesos y 
permitir la realización de eventos memorables.

A continuación se muestra una serie de Aplicaciones Web con funcionalidades parecidas a Enterprise Event Solutions:

\begin{itemize}
    \item \textbf{Eventbrite} es una plataforma versátil para la creación y gestión de eventos. Permite a los usuarios vender entradas online 
    con opciones de precios y tarifas personalizables, así como gestionar el registro y la participación de los asistentes. Además, 
    ofrece herramientas de marketing integradas para promocionar los eventos y análisis de datos detallados sobre la asistencia y 
    el rendimiento de los mismos.
    \item \textbf{Wix Events} es una herramienta de creación de páginas de 
    eventos personalizadas dentro del sitio web de Wix. Ofrece funcionalidades para gestionar el registro de los asistentes y 
    proporciona herramientas de marketing para promocionar los eventos. Además, permite a los usuarios realizar un análisis 
    detallado de la asistencia y el rendimiento del evento a través de datos recopilados en la plataforma.
    \item \textbf{Zoom Events} es una plataforma diseñada para la creación y gestión de eventos virtuales e híbridos, como webinars, 
    reuniones y conferencias. Permite interactuar con los participantes a través de funciones como encuestas, preguntas y respuestas, 
    y chat en vivo. Además, ofrece integraciones con otras herramientas de Zoom para una experiencia más completa y proporciona análisis 
    de datos sobre la asistencia y el rendimiento del evento.
\end{itemize}

El valor añadido distintivo de \textbf{Enterprise Event Solutions}, reside en el enfoque centrado en el usuario y la excepcional 
facilidad de uso. Se ha diseñado la plataforma desde sus cimientos con la premisa de que la accesibilidad y la usabilidad son fundamentales 
para garantizar una experiencia óptima para todos los usuarios, independientemente de su nivel de familiaridad con la tecnología.

Lo que distingue a \textbf{Enterprise Event Solutions} es la capacidad para llegar a usuarios de todos los niveles de habilidad tecnológica. 
Reconozco que no todos los usuarios están familiarizados con las últimas tecnologías, y es por eso que he priorizado la accesibilidad en el 
diseño de mi plataforma. Incluso aquellos que pueden sentirse desactualizados en términos de tecnología encontrarán que \textbf{EVS} es fácil 
de entender y utilizar, gracias a una interfaz intuitiva que les ayuda a navegar por todas las funcionalidades sin dificultad.

